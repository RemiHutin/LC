\documentclass[a4paper,11pt]{article}
\usepackage[french,english]{babel} 
\usepackage{microtype}

\usepackage[utf8]{inputenc}
\usepackage{array}
\usepackage{amsmath} 
\usepackage{amssymb}
\usepackage{amsfonts}
\usepackage{amsthm}
\usepackage{fullpage}%
\usepackage[T1]{fontenc}%

\usepackage{graphicx}%
\usepackage{url}%
\usepackage{abstract}%

\usepackage{mathpazo}%
\parskip=0.5\baselineskip


\title{Binary Decision Diagrams}
\author{Rémi Hutin \& Joshua Peignier}
\date{6 Mai 2016}


\begin{document}
\maketitle

	
	\section{BDD et forme normale if-then-else}

		\paragraph{Question 1}
		Soient $\varphi$ une formule, $x$ une variable et $V$ une valuation.\newline
		Remarquons que, par définition, $\varphi\uparrow^{x} \equiv (x \wedge \varphi[1/x]) \vee (\neg x \wedge \varphi[0/x])$.
		
		Voici donc la table de vérité de $\varphi\uparrow^{x}$ en fonction des valeurs de $x$ et de $\varphi$ pour chaque valuation (on ne remplit que les cases nécessaires).\newline
		
		\begin{tabular}{|c|c|c|c|c|c|c|}
		\hline
		$x$ & $\varphi$ & $\varphi[1/x]$ &  $x \wedge \varphi[1/x]$ & $\varphi[0/x]$ & $\neg x \wedge \varphi[0/x]$ & $\varphi\uparrow^{x}$ \\
		\hline
		0 & 0 &   & 0 & 0 & 0 & 0 \\
		\hline
		0 & 1 &   & 0 & 1 & 1 & 1 \\
		\hline
		1 & 0 & 0 & 0 &   & 0 & 0 \\
		\hline
		1 & 1 & 1 & 1 &   & 0 & 1 \\
		\hline
		\end{tabular}
		
		Il suffit alors de remarquer que les colonnes de $\varphi$ et $\varphi\uparrow^{x}$ sont égales.
		
		\paragraph{Question 2}
		On va procéder par récurrence sur le nombre de variables intervenant dans $\varphi$. Notre hypothèse de récurrence est la propriété $P(n)$ : "toute formule de la logique propositionnelle contenant exactement $n$ variables est équivalente à une formule INF".
	
		Si $\varphi$ ne contient qu'une unique variable $x$, alors on peut directement calculer les valeurs de $\varphi[1/x]$ et de $\varphi[0/x]$, et remplacer celles-ci par leurs valeurs respectives, $0$ ou $1$.
		On peut alors utiliser la question précédente pour écrire que $\varphi \equiv \varphi\uparrow^{x} = x \rightarrow \varphi[1/x],\varphi[0/x]$.
		
		Si on se donne maintenant $n \in \mathbb{N}$ et qu'on suppose que toute formule à $n$ variables est équivalente à une formule INF, alors soit $\varphi$ une formule à $n+1$ variables. Soit $x$ une des variables de $\varphi$, choisie arbitrairement. Remarquons que $\varphi[0/x]$ et $\varphi[1/x]$ sont des formules à $n$ variables. On peut donc leur appliquer l'hypothèse de récurrence. Alors il existe des formules INF $\psi_0$ et $\psi_1$ telles que $\varphi[0/x] \equiv \psi_0$ et $\varphi[1/x] \equiv \psi_1$. Il suffit alors d'écrire que $\varphi \equiv x \rightarrow \psi_1, \psi_0$. L'hypothèse de récurrence permet d'assurer que $\psi_1$ et $\psi_0$ ne sont écrites qu'à l'aide de formules INF et de constantes. La formule que nous venons de construire respecte aussi cette propriété. D'où on déduit l'hérédité.
\end{document}